\chapter{Statements}
In contrast to expressions, \emph{Statements} in ABS are not
evaluated to a value. If one wants to assign a value to statements
it would be the \absinline{Unit} value.

\begin{abssyntax}
\NT{Statement}    \defn \NT{CompoundStmt}
               \defc  \NT{VarDeclStmt}
               \defc  \NT{AssignStmt}
               \defc  \NT{AwaitStmt}
               \defc  \NT{SuspendStmt}
               \defc  \NT{SkipStmt}
               \defc  \NT{AssertStmt}
               \defc  \NT{ReturnStmt}
               \defc  \NT{ExpStmt}\\
\NT{CompoundStmt} \defn \NT{Block}
               \defc  \NT{IfStmt}
               \defc  \NT{WhileStmt}
\end{abssyntax}

\section{Block}
A block consists of a sequence of statements and defines a name scope for variables.

% \subsubsection{Type Checking}
% A block is type-correct if all its statements are type correct.


\begin{abssyntax}
\NT{Block} \defn \TRS{\{}\ \MANY{\NT{Statement}}\ \TRS{\}}
\end{abssyntax}

\section{If Statement}

\begin{abssyntax}
\NT{IfStmt} \defn \TR{if}\ \TRS{(} \NT{PureExp} \TRS{)}\ \NT{Stmt} \OPTG{\TR{else}\ \NT{Stmt}}
\end{abssyntax}

\begin{absexample}
if (5 < x) {
  y = 6;
} else {
  y = 7;
}

if (True)
  x = 5;
\end{absexample}

\section{While Statement}

\begin{abssyntax}
\TR{while}\ \TRS{(} \NT{PureExp} \TRS{)}\ \NT{Stmt}
\end{abssyntax}

\begin{absexample}
while (x < 5)
   x = x + 1;
\end{absexample}

\section{Variable Declaration Statements}
A variable declaration statement is used to declare variables.

\begin{abssyntax}
\NT{VarDeclStmt} \defn \NT{TypeName}\ \NT{Identifier}\ \OPTG{\TRS{=}\ \NT{Exp}}\ \TRS{;}
\end{abssyntax}

A variable has an optional \emph{initialization expression} for defining the initial value of the variable. The initialization expression is \emph{mandatory} for variables of data types.
It can be left out only for variables of reference types, in which case the variable is initialized with \absinline{null}.

\begin{absexample}
Bool b = True;
\end{absexample}

\section{Assign Statement}
The \emph{Assign Statement} assigns a value to a variable or a field.

\begin{abssyntax}
\NT{AssignStmt} \defn \NT{Variable}\ \TRS{=}\ \NT{PureExp}\ \TRS{;}
             \defc  \NT{FieldAccess}\ \TRS{=}\ \NT{PureExp}\ \TRS{;}
\end{abssyntax}

\begin{absexample}
this.f = True;
x = 5;
\end{absexample}


\section{Await Statement}
\emph{Await Statements} suspend the current task until the given
\emph{guard} is true~\cite{johnsen10fmco}. The task will not be
suspended if the guard is already initially true.  While the task is
suspended, other tasks within the same COG can be activated.  Await
statements are also called \emph{scheduling points}, because they are
the only source positions where a task may become suspended and other
tasks of the same COG can be activated.


\begin{abssyntax}
\NT{AwaitStmt}  \defn \TR{await}\ \NT{Guard}\ \TRS{;}\\
\NT{Guard}      \defn \NT{ClaimGuard}
             \defc  \NT{PureExp}
             \defc  \NT{Guard}\ \TRS{\&}\ \NT{Guard}\\
\NT{ClaimGuard} \defn \NT{Variable}\ \TRS{?}
             \defc  \NT{FieldAccess}\ \TRS{?}
\end{abssyntax}

\begin{absexample}
Fut<Bool> f = x!m();
await f?;
await this.x == True;
await f? & this.y > 5;
\end{absexample}

\section{Suspend Statement}
A \emph{Suspend Statement} causes the current task to be suspended.

\begin{abssyntax}
\NT{SuspendStmt} \defn \TR{suspend}\ \TRS{;}
\end{abssyntax}

\begin{absexample}
suspend;
\end{absexample}

\section{Skip Statement}
The \emph{Skip Statement} is a statement that does nothing.

\begin{abssyntax}
\NT{SkipStmt} \defn \TR{skip}\ \TRS{;}
\end{abssyntax}

\section{Assert Statement}\label{sec:abs:assert}
An \emph{Assert Statement} is a statement for asserting certain conditions.

\begin{abssyntax}
\NT{AssertStmt} \defn \TR{assert}\ \NT{PureExp}\ \TRS{;}
\end{abssyntax}

\begin{absexample}
assert x != null;
\end{absexample}

\section{Return Statement}
A \emph{Return Statement} defines the return value of a method.
A return statement can only appear as a last statement in a method body.

\begin{abssyntax}
\NT{ReturnStmt} \defn \TR{return}\ \NT{PureExp}\ \TRS{;}
\end{abssyntax}

\begin{absexample}
return x;
\end{absexample}

\section{Expression Statement}
An \emph{Expression Statement} is a statement that only consists of a single expression. Such statements are only executed for the effect of the expression.

\begin{abssyntax}
\NT{ExpStmt} \defn \NT{Exp}\ \TRS{;}
\end{abssyntax}

\begin{absexample}
new C(x);
\end{absexample}
