 \chapter{User-defined Schedulers}\label{ch:schedulers}
  This section describes how to specify the scheduling of processes
  beyond the basic non-deterministic scheduler.  Processes are scheduled
  per cog.  There are two places where schedulers can be specified:
\begin{enumerate}
\item At a class definition.  All cogs created via an object of the
  class will use the specified scheduling function.
\item As an annotation to the \absinline{new} statement.  This overrides
  any scheduler specified at class level.
\end{enumerate}
  Class-specific schedulers are implemented on the Maude backend and are
  under development on the dynamic Java backend.

\section{Language Elements}

  All identifiers described in the following are contained in the module
  \absinline{ABS.Scheduler}.

  A \emph{scheduler} is a user-defined ABS function that chooses one process
  from a list of processes.  A scheduler will be called with a list of
  processes which are eligible for execution on a specific object (and
  possibly some part of the object's state, see below) and returns the
  process that should execute on that object next.  The user cannot
  currently influence which object is chosen to run by the cog.

\subsection{The \absinline{Process} Datatype}

\begin{absexample} 
data Pid;
data Process = Process(Pid pid, String method, Time arrival,
    Duration cost, Duration procdeadline, Time start, Time finish, 
    Bool crit, Int value);
\end{absexample} 

   \absinline{method} contains the name of the method the process is running,
   \absinline{arrival} the time when the call arrived at the object,
   \absinline{cost} the specified cost of the process (or \absinline{0} when no cost
   annotation given), \absinline{procdeadline} is the deadline (or
   \absinline{InfDuration} when no deadline given at the call site).
   \absinline{start} and \absinline{finish} are bookkeeping values, and \absinline{crit}
   and \absinline{value} are currently unused but will contain criticality
   information and a user-defined priority for the process.

   Note that it is not possible to create a value of type \absinline{Process}
   from within ABS itself, since the datatype \absinline{Pid} has no
   constructor.

\subsection{Defining a Scheduler}

   A scheduler takes a list of \absinline{Process} values and returns one of
   them:

\begin{absexample} 
def Process randomscheduler(List<Process> queue) =
  nth(queue, random(length(queue)));
\end{absexample} 

   This scheduler, defined in the \absinline{ABS.Scheduler} module, chooses a
   random process from the list.  It can be used for
   Monte-Carlo-simulations.

\begin{absexample} 
def Process defaultscheduler(List<Process> queue) = head(queue);
\end{absexample} 

   This is the default scheduler which is used by any class without an
   annotation specifying a different scheduler.

\subsection{The Scheduler Annotation}

   The class annotation \absinline{Scheduler} specifies that objects of a given
   class should use a non-default scheduler.

\begin{absexample} 
[Scheduler: randomscheduler(queue)]
class Test { ... }
\end{absexample} 

The special parameter name \absinline{queue} is a ``magic'' parameter that at each invocation of
   the scheduler function contains the process queue of an object,
   lifted into an ABS value of type \absinline{List<Process>}.

   Scheduling functions can have other parameters as well.  In this
   case, the scheduler will be invoked with values from the current
   object state.  The annotation specifies which fields of the object to
   pass for which parameter of the scheduler:

\begin{absexample} 
def Process experimental_scheduler(List<Process> queue, Int state) = ...

[Scheduler: experimental_scheduler(queue, attribute)]
class Test {
  Int attribute = 0;
  ...
}
\end{absexample} 

   Schedulers specified at class level can be overriden when creating a
   new cog.  Here a cog is created with an object of class \absinline{Test}
   but with the default scheduler:

\begin{absexample} 
  [Scheduler: defaultscheduler(queue)] new Test();
\end{absexample} 



\section{Future Work}
\begin{itemize}
\item Parameters to the \absinline{Scheduler} annotation are not currently
    type-checked.

\item User-defined priorities via the \absinline{value} process member are not
    yet implemented

\item A true \absinline{crit} flag should lead to process failure when the
    deadline is exceeded.  This can be implemented once ABS has a
    failure model.
\end{itemize}